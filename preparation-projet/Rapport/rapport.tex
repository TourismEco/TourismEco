\documentclass[mstat,12pt]{unswthesis}

\usepackage{color}
\usepackage{fancyvrb}
\newcommand{\VerbBar}{|}
\newcommand{\VERB}{\Verb[commandchars=\\\{\}]}
\DefineVerbatimEnvironment{Highlighting}{Verbatim}{commandchars=\\\{\}}
% Add ',fontsize=\small' for more characters per line
\usepackage{framed}
\definecolor{shadecolor}{RGB}{248,248,248}
\newenvironment{Shaded}{\begin{snugshade}}{\end{snugshade}}
\newcommand{\AlertTok}[1]{\textcolor[rgb]{0.94,0.16,0.16}{#1}}
\newcommand{\AnnotationTok}[1]{\textcolor[rgb]{0.56,0.35,0.01}{\textbf{\textit{#1}}}}
\newcommand{\AttributeTok}[1]{\textcolor[rgb]{0.13,0.29,0.53}{#1}}
\newcommand{\BaseNTok}[1]{\textcolor[rgb]{0.00,0.00,0.81}{#1}}
\newcommand{\BuiltInTok}[1]{#1}
\newcommand{\CharTok}[1]{\textcolor[rgb]{0.31,0.60,0.02}{#1}}
\newcommand{\CommentTok}[1]{\textcolor[rgb]{0.56,0.35,0.01}{\textit{#1}}}
\newcommand{\CommentVarTok}[1]{\textcolor[rgb]{0.56,0.35,0.01}{\textbf{\textit{#1}}}}
\newcommand{\ConstantTok}[1]{\textcolor[rgb]{0.56,0.35,0.01}{#1}}
\newcommand{\ControlFlowTok}[1]{\textcolor[rgb]{0.13,0.29,0.53}{\textbf{#1}}}
\newcommand{\DataTypeTok}[1]{\textcolor[rgb]{0.13,0.29,0.53}{#1}}
\newcommand{\DecValTok}[1]{\textcolor[rgb]{0.00,0.00,0.81}{#1}}
\newcommand{\DocumentationTok}[1]{\textcolor[rgb]{0.56,0.35,0.01}{\textbf{\textit{#1}}}}
\newcommand{\ErrorTok}[1]{\textcolor[rgb]{0.64,0.00,0.00}{\textbf{#1}}}
\newcommand{\ExtensionTok}[1]{#1}
\newcommand{\FloatTok}[1]{\textcolor[rgb]{0.00,0.00,0.81}{#1}}
\newcommand{\FunctionTok}[1]{\textcolor[rgb]{0.13,0.29,0.53}{\textbf{#1}}}
\newcommand{\ImportTok}[1]{#1}
\newcommand{\InformationTok}[1]{\textcolor[rgb]{0.56,0.35,0.01}{\textbf{\textit{#1}}}}
\newcommand{\KeywordTok}[1]{\textcolor[rgb]{0.13,0.29,0.53}{\textbf{#1}}}
\newcommand{\NormalTok}[1]{#1}
\newcommand{\OperatorTok}[1]{\textcolor[rgb]{0.81,0.36,0.00}{\textbf{#1}}}
\newcommand{\OtherTok}[1]{\textcolor[rgb]{0.56,0.35,0.01}{#1}}
\newcommand{\PreprocessorTok}[1]{\textcolor[rgb]{0.56,0.35,0.01}{\textit{#1}}}
\newcommand{\RegionMarkerTok}[1]{#1}
\newcommand{\SpecialCharTok}[1]{\textcolor[rgb]{0.81,0.36,0.00}{\textbf{#1}}}
\newcommand{\SpecialStringTok}[1]{\textcolor[rgb]{0.31,0.60,0.02}{#1}}
\newcommand{\StringTok}[1]{\textcolor[rgb]{0.31,0.60,0.02}{#1}}
\newcommand{\VariableTok}[1]{\textcolor[rgb]{0.00,0.00,0.00}{#1}}
\newcommand{\VerbatimStringTok}[1]{\textcolor[rgb]{0.31,0.60,0.02}{#1}}
\newcommand{\WarningTok}[1]{\textcolor[rgb]{0.56,0.35,0.01}{\textbf{\textit{#1}}}}


%%%%%%%%%%%%%%%%%%%%%%%%%%%%%%%%%%%%%%%%%%%%%%%%%%%%%%%%%%%%%%%%%%
% 
% OK...Now we get to some actual input.  The first part sets up
% the title etc that will appear on the front page
%
%%%%%%%%%%%%%%%%%%%%%%%%%%%%%%%%%%%%%%%%%%%%%%%%%%%%%%%%%%%%%%%%%

\title{Groupe Écotourisme\\[0.5cm]Rapport de projet}

\authornameonly{BRANSOLLE Line, GILIBERT Rémy, GONÇALVES
Hugo, MOHAMEDATNI Aya, SÉNÉCAILLE Cassandra, Triozon Lucas }

\author{\Authornameonly}

\copyrightfalse
\figurespagefalse
\tablespagefalse

%%%%%%%%%%%%%%%%%%%%%%%%%%%%%%%%%%%%%%%%%%%%%%%%%%%%%%%%%%%%%%%%%
%
%  And now the document begins
%  The \beforepreface and \afterpreface commands puts the
%  contents page etc in
%
%%%%%%%%%%%%%%%%%%%%%%%%%%%%%%%%%%%%%%%%%%%%%%%%%%%%%%%%%%%%%%%%%%


%%%%%%%%%%%%%%%%%%%%%%%%%%%%%%%%%%%%%%%%%%%%%%%%%%%%%%%%%%%%%%%%%%%%%%%
%
%  A small sample UNSW Coursework Masters thesis file.
%  Any questions to Ian Doust i.doust@unsw.edu.au and/or Gery Geenens ggeenens@unsw.edu.au
%
%%%%%%%%%%%%%%%%%%%%%%%%%%%%%%%%%%%%%%%%%%%%%%%%%%%%%%%%%%%%%%%%%%%%%%%
%
%  The first part pulls in a UNSW Thesis class file.  This one is
%  slightly nonstandard and has been set up to do a couple of
%  things automatically
%
 
%%%%%%%%%%%%%%%%%
%% Precisely one of the next four lines should be uncommented.
%% Choose the one which matches your degree, uncomment it, and comment out the other two!
%\documentclass[mfin,12pt]{unswthesis}    %%  For Master of Financial Mathematics 
%\documentclass[mmath,12pt]{unswthesis}   %%  For Master of Mathematics
%\documentclass[mstat,12pt]{unswthesis}  %%  For Master of Statistics
%%%%%%%%%%%%%%%%%



\linespread{1}
\usepackage{amsfonts}
\usepackage{amssymb}
\usepackage{amsthm}
\usepackage{latexsym,amsmath}
\usepackage{graphicx}
\usepackage{afterpage}
\usepackage[colorlinks]{hyperref}
\hypersetup{
     colorlinks=true,
     linkcolor=blue,
     filecolor=blue,
     citecolor= black,      
     urlcolor=cyan,
     }
\usepackage{textcomp}
\usepackage{longtable}
\usepackage{booktabs}
\usepackage{float}
\let\origfigure\figure
\let\endorigfigure\endfigure
\renewenvironment{figure}[1][2] {
    \expandafter\origfigure\expandafter[H]
} {
    \endorigfigure
}
\usepackage[T1]{fontenc}
\usepackage{ragged2e}
\def\tightlist{}



%
%  Macros - some of these are in plain TeX (gasp!)
%
\newcommand\blankpage{%
    \null
    \thispagestyle{empty}%
    \addtocounter{page}{-1}%
    \newpage}


%%%%%%%%%%%%%%%%%%%%%%%%%%%%%%%%%%%%%%%%%%%%%%%%%%%%%%%%%%%%%%%%%%
%
%  If you've got some funny special words that LaTeX might not
% hyphenate properly, you can give it a helping hand:
%

\hyphenation{Mar-cin-kie-wicz Rade-macher}


\newlength{\cslhangindent}
\setlength{\cslhangindent}{1.5em}
\newlength{\csllabelwidth}
\setlength{\csllabelwidth}{3em}
\newenvironment{CSLReferences}[2] % #1 hanging-ident, #2 entry spacing
 {% don't indent paragraphs
  \setlength{\parindent}{0pt}
  % turn on hanging indent if param 1 is 1
  \ifodd #1 \everypar{\setlength{\hangindent}{\cslhangindent}}\ignorespaces\fi
  % set entry spacing
  \ifnum #2 > 0
  \setlength{\parskip}{#2\baselineskip}
  \fi
 }%
 {}
\usepackage{calc} % for \widthof, \maxof
\newcommand{\CSLBlock}[1]{#1\hfill\break}
\newcommand{\CSLLeftMargin}[1]{\parbox[t]{\maxof{\widthof{#1}}{\csllabelwidth}}{#1}}
\newcommand{\CSLRightInline}[1]{\parbox[t]{\linewidth}{#1}}
\newcommand{\CSLIndent}[1]{\hspace{\cslhangindent}#1}


%% Preambule de Line
\usepackage{fancyhdr}
\setlength{\headheight}{12.80502pt} % Ajustement recommandé
\addtolength{\topmargin}{-0.11499pt} % Ajustement recommandé
\pagestyle{fancy}
\usepackage{tabularx} 
\usepackage{graphicx}
\usepackage{xcolor}
\usepackage{hyperref}
\hypersetup{
  colorlinks=true,
  linkcolor=blue,
  urlcolor=blue,
  citecolor=blue,
  anchorcolor=blue,
  filecolor=blue
}

\setlength{\parskip}{\medskipamount}  % Ajoute un espace vertical entre les paragraphes

\lhead{}
\chead{}
\rhead{}
\lfoot{}
\cfoot{}
\rfoot{}
\renewcommand{\headrulewidth}{0pt}






\renewcommand{\contentsname}{Table des matières}

\renewcommand{\chaptername}{Chapitre}
\usepackage{changepage}
\usepackage{fancyhdr}
\pagestyle{fancy}
\fancyfootoffset{-2cm}




\begin{document}
\begin{adjustwidth}{-2cm}{}

\beforepreface

%\afterpage{\blankpage}


% Acknowledgements are optional


\prefacesection{Remerciements}

{\bigskip}Nos plus sincères remerciements vont à notre encadrant
pédagogique pour les conseils avisés sur notre travail.\\[1cm] 

{\bigskip\bigskip\bigskip\noindent} 2023-12-22.

%\afterpage{\blankpage}

% Abstract

\prefacesection{Résumé}

résume du projet là

%\afterpage{\blankpage}


\afterpreface





%%%%%%%%%%%%%%%%%%%%%%%%%%%%%%%%%%%%%%%%%%%%%%%%%%%%%%%%%%%%%%%%%%
%
% Now we can start on the first chapter
% Within chapters we have sections, subsections and so forth
%
%%%%%%%%%%%%%%%%%%%%%%%%%%%%%%%%%%%%%%%%%%%%%%%%%%%%%%%%%%%%%%%%%%



%%%%%%%%%%%%%%%%%%%%%%%%%%%%%%%%%%%%%

%\afterpage{\blankpage}


\begin{Shaded}
\begin{Highlighting}[]
\KeywordTok{SELECT} \KeywordTok{id}
\KeywordTok{FROM}\NormalTok{ pays}
\end{Highlighting}
\end{Shaded}

\begin{longtable}[]{@{}l@{}}
\caption{Pays}\tabularnewline
\toprule\noalign{}
id \\
\midrule\noalign{}
\endfirsthead
\toprule\noalign{}
id \\
\midrule\noalign{}
\endhead
\bottomrule\noalign{}
\endlastfoot
AD \\
AE \\
AG \\
AI \\
AL \\
AM \\
AO \\
AR \\
AS \\
AT \\
\end{longtable}

\hypertarget{introduction}{%
\chapter{Introduction}\label{introduction}}

Lorsque les vacances approchent, l'envie de quitter la routine
quotidienne s'accompagne souvent de la recherche d'une destination
idéale. Pour cette étape, plusieurs critères peuvent entrer en compte:
le prix, évidemment, mais aussi le moyen de transport, l'hébergement,
les activités ou encore l'empreinte carbone.

\par

Le tourisme est une activité humaine qui a des impacts multiples et
complexes sur l'environnement, l'économie et la société. Selon
l'Organisation mondiale du tourisme (OMT), le tourisme international a
atteint 1,4 milliard d'arrivées en 2018, soit une croissance de 6 \% par
rapport à l'année précédente. Cette tendance à la hausse devrait se
poursuivre dans les prochaines années, avec une projection de 1,8
milliard d'arrivées en 2030.

\par

En effet, face au changement climatique, de plus en plus de personnes
souhaitent faire attention à impacter le moins possible l'environnement
lors de leurs déplacements. Pendant longtemps, le tourisme a été
appréhendé uniquement au niveau de ses retombées économiques. En 2018,
la consommation touristique représentait 7,4 \% du PIB en France, avec
une croissance dynamique, portée principalement par les visiteurs
étrangers.

\par

Dans le même temps, le secteur engendre de nombreuses pressions sur
l'environnement, et en particulier sur le climat. Selon des données, le
tourisme génère actuellement 11 \% des émissions de gaz à effet de serre
en France. À l'heure où le monde ressent l'urgence de prendre des
mesures significatives face aux défis environnementaux, le secteur du
tourisme émerge comme l'un des contributeurs majeurs aux émissions de
gaz à effet de serre.

\par

Cependant, il n'existe pas de solution simple et universelle pour
concilier tourisme et développement durable. Chaque destination présente
des spécificités et des défis qui nécessitent une analyse approfondie et
une adaptation permanente. C'est pourquoi nous avons créé Écotourisme,
une plateforme en ligne innovante qui propose aux voyageurs une
navigation personnalisée et responsable. Écotourisme se positionne comme
la réponse à cette quête, prenant en compte une variété de critères,
dont le prix, le moyen de transport, et surtout, l'empreinte carbone.

\par

Écotourisme a été conçu pour répondre aux trois besoins principaux de
nos utilisateurs: la réduction des dépenses, la simplification du
processus de voyage, et la promotion de la responsabilité
environnementale. Notre projet joue également un rôle de comparateur
statistique. En effet, en collectant des données, nous présentons des
statistiques couvrant de nombreuses années et divers indicateurs liés à
l'économie, à l'écologie et au tourisme. À l'aide de graphiques et d'un
score ``Ecotourisme'' attribué à chaque pays, les utilisateurs ont accès
à des données globales, continentales et nationales. De plus, il est
possible de comparer les données statistiques entre deux pays et choisir
celui qui correspond le mieux à vos attentes et à vos valeurs. Que vous
soyez un voyageur économe, un éco-voyageur engagé, ou simplement un
passionné soucieux de minimiser son impact sur l'environnement,
Écotourisme est votre solution.

\par

Ce qui distingue fondamentalement notre plateforme est son approche
intégrée. Contrairement à d'autres services qui se concentrent
uniquement sur l'aspect économique ou environnemental, Écotourisme
fusionne ces deux dimensions pour offrir une solution complète. Ce
rapport a pour but de présenter notre projet Écotourisme, une plateforme
qui propose aux voyageurs une offre personnalisée et responsable, basée
sur des données statistiques et des critères économiques et
environnementaux.

\hypertarget{sujet}{%
\chapter{Sujet}\label{sujet}}

\hypertarget{base-de-donnuxe9es}{%
\chapter{Base de données}\label{base-de-donnuxe9es}}

\hypertarget{duxe9veloppement-et-outils}{%
\chapter{Développement et outils}\label{duxe9veloppement-et-outils}}

\hypertarget{planning}{%
\chapter{Planning}\label{planning}}

\textbf{Rapport d'Avancement du Projet}

\hypertarget{introduction-1}{%
\section{Introduction}\label{introduction-1}}

La planification d'un projet demeure un élément crucial pour assurer une
progression efficace et la réussite globale. Au cours de cette période,
notre équipe s'est concentrée sur la mise en œuvre et le suivi rigoureux
d'un plan établi, en prenant en compte les ajustements nécessaires pour
répondre aux éventuels défis rencontrés. L'outil MindView a été intégré
dans notre processus de planification pour fournir une visualisation
claire des tâches et des dépendances. Les diagrammes générés ont été
d'une grande valeur pour la communication interne et ont facilité la
compréhension des membres de l'équipe sur les différentes phases du
projet. Ainsi il nous servira à illustrer ce rapport.

\hypertarget{global}{%
\section{Global}\label{global}}

Au cours des premiers cycles du projet, l'équipe a consacré une
attention particulière à plusieurs phases clés, conformément au planning
initial. Les activités suivantes ont été menées avec succès, contribuant
à l'avancement global du projet:

\begin{itemize}
\item
  \textbf{Description du Projet (13 jours):} Une période substantielle a
  été allouée à la compréhension approfondie des exigences du projet.
  Les sessions de brainstorming et les réunions de travail ont permis de
  définir clairement les objectifs, les livrables attendus, et les
  attentes des parties prenantes.
\item
  \textbf{Modélisation UML (8 jours):} Dans cette phase, l'équipe a
  consacré huit jours à la création de modèles UML détaillés. Ces
  modèles ont servi de fondement conceptuel, facilitant la visualisation
  des différentes composantes du système et établissant des relations
  claires entre celles-ci.
\item
  \textbf{Élaboration du Canevas (8 jours):} Une attention particulière
  a été accordée à l'élaboration du canevas du projet, fournissant une
  représentation visuelle de l'architecture globale. Cette étape a
  permis de valider les besoins du projet avant de passer à des phases
  plus détaillées.
\item
  \textbf{Conception MCD/MOD (8 jours):} Huit jours ont été alloués à la
  conception du Modèle Conceptuel de Données (MCD) et du Modèle
  Organisationnel de Données (MOD). Ces modèles ont permis de définir la
  structure des données et les relations, guidant ainsi le développement
  ultérieur.
\item
  \textbf{Maquettage avec Figma (13 jours):} Comme décrit plus tôt la
  phase de maquettage avec Figma a occupé treize jours, pendant lesquels
  l'équipe a créé des maquettes interactives et visuelles du projet. Ces
  maquettes ont été utilisées pour obtenir des retours précoces des
  parties prenantes et ajuster les éléments de conception en
  conséquence. Même si le temps de réalisation en nombre de jour est
  similaire aux autres tâches, celle-ci nous a demandé un investissement
  plus important, par la quantité de travail qu'elle représente.
  Élaborer un diagramme n'est pas comparable à imaginer une dizaine de
  pages Web, et le résultat a été obtenu par une concentration
  particulière de tous nos moyens sur cet objectif. Ainsi comme monter
  sur l'image plus tôt la mise en place avant codage nous a pris 1 mois
  complet afin de cerner l'entièreté des enjeux de ce projet. On
  remarque aussi que l'organisation de notre groupe nous a permis de
  travailler en différé sur ces taches et ainsi de gagner un temps
  précieux.
\end{itemize}

\hypertarget{partie-base-de-donnuxe9es}{%
\section{Partie Base de Données}\label{partie-base-de-donnuxe9es}}

\medskip

La gestion évolutive de la base de données a été une composante cruciale
du développement du projet, consolidant progressivement les données
touristiques et économiques pour offrir une expérience riche et
complète. Voici un aperçu détaillé des différentes étapes de
développement dans la partie base de données:

\begin{itemize}
\tightlist
\item
  \textbf{13/09 - 21/09:} Première Version et Réflexions, Données
  Touristiques et Pays (8 jours) Au cours de cette phase initiale, la
  première version de la base de données a été élaborée, avec un accent
  particulier sur les données touristiques et les informations
  essentielles sur chaque pays. Des réflexions approfondies ont orienté
  la conception initiale pour garantir une fondation solide.
\item
  \textbf{21/09 - 30/09:} Recherche et Ajout de Données Complémentaires,
  Économie, Écologie et Sûreté, Mise en Place des Clés Étrangères (9
  jours) La base de données a été enrichie par des données
  complémentaires, notamment des informations sur l'économie, l'écologie
  et la sûreté des pays. Les clés étrangères ont été soigneusement mises
  en place pour établir des relations significatives entre les
  différentes tables.
\item
  \textbf{02/10 - 09/10:} Réflexion et Ajouts Mineurs: PIB/hab,
  Continents, Emojis (8 jours) Cette période a été consacrée à des
  réflexions approfondies, conduisant à des ajouts mineurs mais
  significatifs tels que le PIB par habitant, la classification par
  continents et l'introduction d'emojis pour représenter les drapeaux de
  nos pays afin d'obtenir une expérience utilisateur plus visuelle et
  intuitive.
\item
  \textbf{09/10 - 11/10:} Fusion des Tables par Thèmes (2 jours) Une
  étape importante a été la fusion des tables par thèmes, améliorant
  ainsi la cohérence et la facilité d'accès aux données. Cette
  restructuration a contribué à simplifier la gestion de la base de
  données et à rendre les requêtes plus efficientes.
\item
  \textbf{07/11 - 12/11:} Ajout des Données sur les Aéroports et les
  Routes Aériennes (6 jours) L'expansion de la base de données a
  continué avec l'intégration de données sur les aéroports et les routes
  aériennes. Cette addition a considérablement enrichi le volet
  logistique du projet, offrant aux utilisateurs des informations
  détaillées sur les infrastructures aériennes. Ajout d'Images pour
  Chaque Pays: La base de données sera étendue pour inclure des images
  associées à chaque pays, améliorant ainsi l'expérience utilisateur en
  offrant une représentation visuelle. Gestion Continue: En
  reconnaissance du caractère dynamique de la base de données, une
  approche agile sera maintenue pour permettre des mises à jour
  régulières en fonction des besoins changeants du projet écotourisme.
\end{itemize}

\hypertarget{carte}{%
\section{Carte}\label{carte}}

\begin{itemize}
\tightlist
\item
  \textbf{02/10 - 10/10:} Première Version (Fond de Carte, Tests) (9
  jours) Durant cette période, l'équipe a concentré ses efforts sur la
  création de la première version du projet. C'était la première fois
  que nous approchions cet outil, et la création de cette version
  initiale visait à explorer les fonctionnalités offertes et à évaluer
  sa pertinence. Cette étape préliminaire était cruciale pour prendre
  une décision éclairée sur l'utilisation continue de l'outil,
  confirmant ainsi notre choix de poursuivre avec amCharts.
\item
  \textbf{10/10 - 11/10:} Comportement au survol (HTML des Mini-Cartes,
  Tests) (2 jours) La seconde phase du développement, centrée sur le
  comportement au survol, s'est basée sur les premiers bandeaux réalisés
  pour la page pays. En utilisant ces éléments comme référence, l'équipe
  a rédigé le code HTML derrière l'affichage des mini-cartes au survol,
  renforçant ainsi la cohérence de l'interface utilisateur.
\item
  \textbf{12/10 - 13/10:} Affichage des Villes, Focus sur un Pays (2
  jours) Ces deux jours ont été consacrés à l'extension des
  fonctionnalités du projet, notamment l'affichage des villes et la
  fonction de mise en évidence d'un pays spécifique. Ces ajouts
  enrichissent l'expérience utilisateur en fournissant des informations
  détaillées sur les différentes régions.
\item
  \textbf{23/10 - 25/10:} Réécriture en JS Pur (3 jours) Une évolution
  majeure a eu lieu avec la réécriture du code en JavaScript pur. Cette
  transition vers un fichier séparé permet de réutiliser la carte dans
  n'importe quelle page souhaitée, anticipant notamment notre future
  page comparateur. Cette modularité renforce l'efficacité du
  développement et la maintenance à long terme.
\item
  25/10: Adaptation au Comparateur (Comportement au Clic pour Deux Pays)
  (1 jour) Pour rendre le projet plus polyvalent, des ajustements ont
  été apportés pour permettre le support de deux pays mis en avant au
  lieu d'un seul. Ces modifications s'alignent avec notre vision
  d'extension future du projet vers une page comparateur, offrant une
  expérience utilisateur plus complète.
\item
  \textbf{15/12 - 17/12:} Réécriture en Orienté Objet + Villes
  Dynamiques au Clic (3 jours) Une refonte majeure a été entreprise en
  adoptant une approche orientée objet. Inspirée par l'architecture POO
  mise en place pour les graphiques, cette transition améliore la
  lisibilité du code, favorise la réutilisation des fonctionnalités, et
  simplifie la gestion des futures évolutions du projet.
\end{itemize}

\hypertarget{page-pays-monde-comparateur}{%
\section{Page Pays/ Monde/
Comparateur}\label{page-pays-monde-comparateur}}

Chaque page, caractérisée par des fonctionnalités graphiques avancées, a
fait l'objet d'un travail en binôme, favorisant la collaboration et la
génération d'idées innovantes. Initialement planifiées sur une plage de
21 à 41 jours, ces périodes ont été recalculées et étendues jusqu'au 22
décembre en raison de contraintes imprévues. (partiels et difficultés
sous-estimées). Page Pays : La page pays a été conçue pour offrir une
expérience approfondie, avec des graphiques détaillés reflétant diverses
données économiques, touristiques et environnementales. Page Monde: La
complexité des graphiques de la page Monde, agrégeant des données de
multiples pays, a demandé une période étendue pour garantir une
représentation précise.

\hypertarget{page-comparateur}{%
\section{Page Comparateur}\label{page-comparateur}}

La page comparateur, mettant en œuvre des graphiques comparatifs entre
deux pays, a exigé une période supplémentaire pour perfectionner les
interactions utilisateur. L'extension a permis d'affiner les
fonctionnalités de comparaison, d'ajuster les détails visuels et de
garantir une expérience utilisateur fluide. Cette recalibration des
délais a été nécessaire pour surmonter les challenges techniques
inhérents à la mise en œuvre de fonctionnalités graphiques avancées.
Malgré l'ajustement des périodes, l'objectif principal a toujours été de
fournir des pages robustes, offrant des représentations visuelles
précises des données tout en maintenant des performances optimales.

\hypertarget{pages-annexes}{%
\section{Pages annexes}\label{pages-annexes}}

La conception et le développement des pages Profil, Inscription,
Connexion, About, Catalogue des Pays, ainsi que la Footer/Navbar ont été
impactés par les ajustements précédents. Ces pages étaient initialement
planifiées entre le 21 novembre et le 3 janvier. Voici un aperçu des
tâches à accomplir au cours de cette période:

\begin{itemize}
\tightlist
\item
  \textbf{Page Profil (18/12 - 15/01):} Cette page centrale accueillera
  les fonctionnalités de gestion de profil utilisateur, offrant une
  expérience personnalisée. La collaboration en binôme se poursuivra
  pour intégrer des fonctionnalités avancées et assurer une parfaite
  harmonie visuelle avec les autres pages.
\item
  \textbf{Pages d'Inscription et de Connexion (18/12 - 15/01):} Ces
  pages cruciales seront revisitées pour garantir une inscription fluide
  et une connexion sécurisée. L'extension de la période permettra de
  prendre en compte les évolutions apportées aux autres parties du site,
  assurant ainsi une intégration homogène.
\item
  \textbf{Page About (18/12 - 15/01):} La page About bénéficiera
  également d'ajustements pour refléter les dernières améliorations
  apportées aux autres pages. La période étendue permettra d'approfondir
  la présentation du projet, de l'équipe et des objectifs globaux.
\item
  \textbf{Catalogue des Pays (18/12 - 15/01):} La création du catalogue
  des pays facilitera la navigation à travers les informations, offrant
  une expérience plus rapide et conviviale. Cette fonctionnalité
  optimisera la recherche d'informations spécifiques sur les différents
  pays / continent présents dans la base de données.
\item
  \textbf{Footer/Navbar (18/12 - 15/01):} Le Footer et la Navbar, seront
  ajusté pour assurer une navigation intuitive sur l'ensemble du site.
  Les liens vers les différentes pages, y compris les nouvelles, seront
  intégrés de manière cohérente. La période allant du 18 décembre au 15
  janvier sera cruciale pour finaliser ces pages, en garantissant une
  intégration homogène des nouvelles fonctionnalités tout en maintenant
  une expérience utilisateur fluide et cohérente sur l'ensemble du site.
  L'approche collaborative en binôme sera maintenue pour maximiser
  l'efficacité et la créativité dans la réalisation de ces dernières
  étapes du projet.
\end{itemize}

\hypertarget{derniuxe8re-pages-et-enjeux-cruciaux}{%
\section{Dernière pages et enjeux
cruciaux}\label{derniuxe8re-pages-et-enjeux-cruciaux}}

Au cours de la période s'étendant du 15 janvier au 29 janvier, les
efforts de développement seront concentrés sur la création des pages
Continent et Calculateur. Parallèlement, un contrôle qualité exhaustif
sera effectué sur l'ensemble des pages du site pour garantir une
expérience utilisateur homogène. Ces pages ont également été affectées
par les ajustements de planning antérieurs.

\begin{itemize}
\tightlist
\item
  \textbf{Page Continent (15/01 - 29/01):} La page Continent a été
  conçue pour offrir une expérience approfondie, avec des graphiques
  détaillés reflétant diverses données économiques, touristiques et
  environnementales, elle est proche de la page pays.
\item
  \textbf{Page Calculateur (15/01 - 29/01):} La page Calculateur,
  réputée pour sa complexité technique, sera spécifiquement développée
  pendant cette période. Le calcul de l'indice, le contrôle qualité des
  résultats, et l'ajustement visuel seront les principales priorités. Un
  travail méticuleux sera effectué pour garantir la précision des
  calculs.
\item
  \textbf{Contrôle Qualité Global (15/01 - 29/01):} Simultanément au
  développement des nouvelles pages, un contrôle qualité complet sera
  réalisé sur l'ensemble du site. Chaque page sera passée en revue pour
  s'assurer de la cohérence des styles CSS, de la précision des
  fonctionnalités, et de la fluidité de l'expérience utilisateur. Des
  ajustements seront apportés si nécessaire. Cette approche holistique,
  combinant le développement des nouvelles pages avec un contrôle
  qualité global, vise à garantir la qualité et la cohérence à tous les
  niveaux du projet. Les ajustements seront apportés en fonction des
  résultats du contrôle qualité pour atteindre un produit final
  répondant aux normes élevées établies par l'équipe. La période du 15
  janvier au 29 janvier sera cruciale pour la finalisation réussie de
  ces étapes.
\end{itemize}

\hypertarget{conclusion-planning}{%
\section{Conclusion planning}\label{conclusion-planning}}

Nous avons rencontré des défis imprévus, notamment des contraintes de
temps dues à la complexité des tâches et des ajustements nécessaires
pour garantir la qualité globale d'écotourisme. Chaque membre de
l'équipe s'est investi dans des domaines spécifiques (page différente
sur fonctionnement de binôme et tache annexe), permettant une approche
spécialisée et une meilleure efficacité. La collaboration en binôme a
renforcé la créativité et la résolution de problèmes, favorisant ainsi
des résultats innovants. À l'approche de la phase finale du projet, nous
restons résolus à surmonter les obstacles restants et à livrer un projet
final répondant aux normes élevées que nous avons établies.

\hypertarget{conclusion}{%
\chapter{Conclusion}\label{conclusion}}

Pour conclure ce rapport, Écotourisme se dresse comme une réponse
novatrice aux défis environnementaux posés par l'industrie du tourisme.
À l'heure où ce secteur contribue de manière significative aux émissions
de gaz à effet de serre, notre plateforme incite à une approche plus
responsable du voyage.

\par

Notre engagement en faveur d'un tourisme responsable se manifeste à
travers une approche intégrée. Écotourisme n'est pas seulement une
plateforme économique, mais un écosystème complet où le plaisir du
voyage coexiste avec la responsabilité écologique. Ce rapport, subdivisé
en plusieurs parties clés, témoigne de notre engagement envers la
transparence. À travers la présentation des objectifs, de la base de
données, du développement, et des outils, nous partageons chaque étape
de notre parcours. Lors de la conception et du développement de notre
plateforme Ecotourisme, certaines contraintes et limitations cruciales
ont été prises en compte pour garantir le succès du projet. Ces
contraintes et limitations guideront notre équipe tout au long du
projet, garantissant que notre plateforme ``Ecotourisme'' répondra aux
normes les plus élevées, tout en restant flexible et innovante dans sa
conception et son développement.

\par

La section dédiée à l'empreinte carbone souligne l'importance cruciale
de comprendre les conséquences environnementales de nos choix de
déplacement. En mettant en lumière les émissions de CO2 associées à
différents trajets, nous avons souhaité éduquer nos utilisateurs et les
inciter à des choix de voyage plus durables.

\par

Écotourisme n'est pas simplement une plateforme de voyage, mais une
invitation à repenser la manière dont nous explorons le monde. Nous
croyons en un avenir où le plaisir du voyage s'aligne avec la
préservation de notre planète.

\par

Ensemble, avec nos utilisateurs, et la communauté mondiale, nous
façonnons un avenir du tourisme plus respectueux, plus vert, et plus
durable. Rejoignez-nous dans cette aventure vers un monde où chaque
voyage compte, non seulement pour nous, mais aussi pour les générations
futures. Voyagez avec conscience. Voyagez avec Écotourisme.





\end{adjustwidth}
\end{document}

